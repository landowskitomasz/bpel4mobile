\documentclass[11pt]{aghdpl}
% \documentclass[en,11pt]{aghdpl}  % praca w języku angielskim
\usepackage[polish]{babel}
%\usepackage[english]{babel}
\usepackage[utf8]{inputenc}

% dodatkowe pakiety
\usepackage{enumerate}
\usepackage{listings}
\lstloadlanguages{TeX}

\lstset{
  literate={ą}{{\k{a}}}1
           {ć}{{\'c}}1
           {ę}{{\k{e}}}1
           {ó}{{\'o}}1
           {ń}{{\'n}}1
           {ł}{{\l{}}}1
           {ś}{{\'s}}1
           {ź}{{\'z}}1
           {ż}{{\.z}}1
           {Ą}{{\k{A}}}1
           {Ć}{{\'C}}1
           {Ę}{{\k{E}}}1
           {Ó}{{\'O}}1
           {Ń}{{\'N}}1
           {Ł}{{\L{}}}1
           {Ś}{{\'S}}1
           {Ź}{{\'Z}}1
           {Ż}{{\.Z}}1
}

%---------------------------------------------------------------------------

\author{Tomasz Landowski}
\shortauthor{T. Landowski}

\titlePL{Analiza możliwości adaptacji środowisk uruchomieniowych dla logiki biznesowej na platformach mobilnych.}
\titleEN{Analysis of possible adaptation or business logic runtimes for mobile platforms.}

\shorttitlePL{Adaptacja środowisk uruchomieniowych dla logiki biznesowej na platformach mobilnych.} % skrócona wersja tytułu jeśli jest bardzo długi
\shorttitleEN{Analysis of possible adaptation or business logic runtimes for mobile platforms.}

\thesistype{Praca dyplomowa magisterska}
%\thesistype{Master of Science Thesis}

\supervisor{dr hab. Grzegorz J. Nalepa}
%\supervisor{Grzegorz J. Nalepa PhD, DSc}

\degreeprogramme{Informatyka}
%\degreeprogramme{Computer Science}

\date{2014}

\department{Katedra Informatyki Stosowanej}
%\department{Department of Applied Computer Science}

\faculty{Wydział Elektrotechniki, Automatyki,\protect\\[-1mm] Informatyki i Inżynierii Biomedycznej}
%\faculty{Faculty of Electrical Engineering, Automatics, Computer Science and Biomedical Engineering}

\acknowledgements{Serdecznie dziękuję...}


\setlength{\cftsecnumwidth}{10mm}

%---------------------------------------------------------------------------
\setcounter{secnumdepth}{4}

\begin{document}

\titlepages
\setcounter{tocdepth}{3}
\tableofcontents
\clearpage

\chapter{Wstęp}
\label{cha:wstep}

Wzrost popularności aplikacji mobilnych głównie ze względu na powszechny dostępu do urządzeń przenośnych takich jak smartphone czy tablet, wpłynął na wzmocnienie ich pozycji na rynku IT. W~ostatnim czasie powstał szereg firm zajmujących się tworzeniem tylko tego rodzaju systemów informatycznych. Przyglądając się współczesnym trendom w IT nie sposób więc nie zwrócić szczególnej uwagi  właśnie na aplikacje mobilne. Mimo wcześniej wspomnianego wzrostu popularności systemy mobilne ciągle wymagają od programistów korzystania z niskopoziomowych bibliotek i narzędzi. Ze względu na rodzaj problemów rozwiązywanych przez tego typu aplikacje może to być częściowo zrozumiałe. Aplikacje mobilne jednak to nie tylko proste programy uruchamiane w ograniczonym sprzętowo środowisku. Twórcy tego rodzaju aplikacji muszą się zmagać z problemami synchronizacji danych, rozproszonej architektury  itd. 
Z drugiej zaś strony wyżej wymienione problemy są już od dawna znane, co więcej posiadają wypracowane schematy ich rozwiązywania. W przypadku aplikacji internetowych powstało mnóstwo frameworków i narzędzi wspierających tworzenie tego rodzaju systemów. Szablonowe rozwiązania pozwoliły natomiast na przeniesienie rozwiązywania problemów na znacznie wyższy poziom niż pisanie kodu aplikacji. Powstały w ten sposób różnego rodzaju systemy uruchomieniowe logiki biznesowej, skalowane i łatwo modyfikowalne. 
Niniejsza praca magisterska jest próbą wykorzystania istniejących środowisk uruchomieniowych dla logiki biznesowej w środowisku aplikacji mobilnych. 

%---------------------------------------------------------------------------

\section{Cel pracy}
\label{sec:celePracy}

Celem niniejszej pracy magisterskiej jest zestawienie technologii  modelowania logiki biznesowej z~aplikacjami mobilnymi. Powinno ono prowadzić do wyciągnięcia wniosków dotyczących sensowności oraz sposobu wykorzystania tych dwóch kompletnie różnych technologii razem. 

%---------------------------------------------------------------------------

\section{Struktura pracy}
\label{sec:strukturaPracy}

W rozdziale ~\ref{cha:bpmVSMobileApplications} została zawarta analiza modelowania procesów biznesowych w kontekście aplikacji mobilnych. Rozdział został podzielony na cztery sekcje, które zawierają kolejno:
\begin{itemize}
\item Sekcja ~\ref{sec:analizaModelowaniaProcesowBiznesowych} - Opis technologii modelowania procesów biznesowych.
\item Sekcja ~\ref{sec:analizaAplikacjiMobilnych} - Opis specyfiki aplikacji na platformy mobilne.
\item Sekcja ~\ref{sec:bpel} - Opis BPEL - jednej z implementacji BPM w środowisku usług sieciowych. 
\item Sekcja ~\ref{sec:bpelVSmobileApp} - Zestawienie BPEL z aplikacjami mobilnymi.
\end{itemize}

Rozdział ~\ref{cha:middleware} jest opisem projektu warstwy pośredniej, której zastosowanie zostało zaproponowane w~rozdziale poprzednim. Rozdział rozpoczyna się od przedstawienia bardzo poglądowych diagramów projektu, by następnie poprzez kolejne fazy projektowania rozwiązana przejść do bardziej szczegółowych diagramów klas i sekwencji. 

Zwieńczeniem pracy magisterskiej jest rozdział ~\ref{cha:example}, który zawiera kompleksowy przykład zastosowania przedstawionego wcześniej rozwiązania. Rozdział zawiera poglądowy diagram BPMN implementowanego procesu biznesowego i opis czterech podsystemów niezbędnych do implementacji tego procesu.

\chapter{Analiza problemu modelowania procesów biznesowych}
\label{cha:analizaModelowaniaProcesowBiznesowych}

Modelowanie procesów biznesowych jest pojęciem bardzo ogólnym, nie zawiera w sobie żadnych szczegółów technicznych, określa jedynie pewne specyficzne podejście do rozwiązywania problemów informatycznych. W podejściu tym podczas analizy, główny nacisk kładziony jest na wyłonienie procesów występujących w analizowanym problemie. Proces jest tutaj rozumiany jako zbiór następujących w określonej kolejności operacji, prowadzących do osiągnięcia konkretnego celu. Najczęściej modelowanie procesów biznesowych jest rozważane w kontekście działania konkretnego przedsiębiorstwa, w którym wyłonienie procesów oraz odpowiednie nimi zarządzenie ma kluczowe znaczenie dla osiągnięcia sukcesu. 

%---------------------------------------------------------------------------

\section{BPM}
\label{sec:bpm}

W literaturze pojęcie modelowania oraz zarządzanie procesami biznesowymi w kontekście przedsiębiorstw występuje pod skrótem BPM, którego angielskie rozwinięcie to Buisness Process Management (pl.  Zarządzanie Procesami Biznesowymi). Idea BPM jest bardzo popularna i szeroko rozwijana w środowisku IT, powstała nawet organizacja pod nazwą \textit{European Association of Buisness Process Managment} zajmująca się rozwojem i promocją BPM.  Na stronie internetowej organizacji znaleźć możemy oficjalną definicje BPM, która mówi, że w skład Zarządzania Procesami Biznesowymi wychodzi:
\begin{itemize}
\item projektowanie,
\item wykonywanie,
\item dokumentacja,
\item pomiar,
\item monitorowanie,
\item kontrola
\end{itemize}
zautomatyzowanych oraz niezautomatyzowanych procesów biznesowych.~\cite{EAOBPMWeb}

W środowisku IT zdarza się, że BPM traktowany jest jako osobna klasa systemów informatycznych, obok ERP, MES, CRM itd. Z punktu widzenia przedstawionej wyżej definicji trudno zgodzić się z takim podejściem, można jednak zauważyć że BPM de facto może zostać wykorzystany do realizacji każdego z wymienionych rodzajów oprogramowania lub posłużyć jako narzędzie integrujące wyżej wymienione klasy systemów, w celu stworzenia globalnego systemu zarządzania przedsiębiorstwem. ~\cite{wiBPMA}

%---------------------------------------------------------------------------

\section{Projektowanie BPM}
\label{sec:projektowanieBPM}

Zdecydowanie jednym z najistotniejszych etapów w tworzeniu systemu opartego o procesy biznesowe jest etap projektowania. Jednym z najbardziej popularnych narzędzi do tego celu jest BPMN (\textit{Buisness Process Modeling Notation}), które doczekało się już dwóch wersji. W kontekście niniejszej pracy magisterskiej notacja BPMN nie jest szczególnie ważna jednak została wspomniana ze względu na jej wykorzystanie w przykładach. Warto zwrócić uwagę  na fakt, że BPMN nie jest w żadnym stopniu implementacją, jest to jedynie sposób wizualizacji procesu za pomocą diagramu w postaci obrazka.

%---------------------------------------------------------------------------

\section{Wykonywanie BPM}
\label{sec:wykonywanieBPM}

Jak zostało wspomniane na samym początku Modelowanie Procesów Biznesowych jest pojęciem ogólnym nie wskazującym konkretnej technologii, która ma posłużyć do wykonania procesów biznesowych. Z definicji można jednak wnioskować, że środowiskiem do wykonywania procesów biznesowych w przedsiębiorstwach na pewno nie powinna być prosta aplikacja desktopowa czy mobilna. Biorąc pod uwagę wachlarz zastosowań BPM nie trudno dojść do wniosku, że najbardziej odpowiednim środowiskiem dla procesów biznesowych są środowiska aplikacji rozproszonych. 
Powyższe stwierdzenie potwierdza przegląd istniejących systemów uruchomieniowych dla procesów biznesowych. W ogromnej większości są to aplikacje webowe. 

%---------------------------------------------------------------------------

\section{Podsumowanie}
\label{sec:podsumowanieBPM}

Z punktu widzenia niniejszej pracy magisterskiej, głównym wnioskiem płynącym z powyższego opisu BPM jest skala problemów rozwiązywanych przez to podejście oraz ich usytuowanie w środowisku aplikacji rozproszonych. 

\chapter{Analiza aplikacji na platformy mobilne}
\label{cha:analizaAplikacjiMobilnych}

Wzrost popularności smartphonów oraz urządzeń przenośnych takich jak tablet wpłynął na powstanie specyficznej klasy systemów informatycznych zwanych aplikacjami mobilnymi. Aplikacje mobilne cechuje ukierunkowanie na rozwiązywanie wąskiego rodzaju problemów przy pomocy urządzenia dostępnego dla użytkownika niemal 24h na dobę. Szczególną popularność zyskały aplikacje mobilne ukierunkowane na użytkowników (TODO jak nazwać użytkownika typu powszechny Jan Kowalski). Coraz większą popularność zyskują również aplikacje biznesowe ukierunkowane np. na wymianę danych pomiędzy pracownikami przedsiębiorstwa w celu rozwiązania konkretnego zadania. 

%---------------------------------------------------------------------------
\section{Rodzaje aplikacji mobilnych}
\label{sec:rodzajeAplikacjiMobilnych}

Ze względu na rodzaj zastosowania aplikacje mobilne można sklasyfikować w następujący sposób:

\begin{itemize}
\item Samodzielne aplikacje -- aplikacje wykorzystujące tylko i wyłącznie lokalne zasoby urządzenia, korzystające z lokalnej bazy danych bez połączenia z systemami zewnętrznym, często nie wymagające połączenie do internetu. Dobrym przykładem takiej aplikacji może być prosty notatki.  
\item Aplikacje klienckie -- są to aplikacje bazujące na komunikacji z systemami zewnętrznymi przy pomocy jakiegokolwiek interfejsu komunikacyjnego,  najczęściej połączenia HTTP. Zazwyczaj aplikacje te wykorzystują również lokalną bazę danych umożliwiając użytkownikom pracę z aplikacji w przypadku braku połączenia z systemem zewnętrznym.
\item Internetowe -- najczęściej strony www, aplikacje  te nie wykorzystują lokalnych zasobów urządzeń mobilnych, ich rolą jest udostępnienie prostego interfejsu użytkownika systemu dla użytkowników aplikacji mobilnych.
\item Gry  -- szczególny przypadek samodzielnych aplikacji ukierunkowanych głównie na wykorzystanie lokalnych zasobów urządzenie w celach rozrywkowych, szczególnie eksploatowana w tego typu aplikacjach jest karta graficzna urządzenia.
\end{itemize}

%---------------------------------------------------------------------------
\section{Tryby online/offline}
\label{sec:trybyAplikacjiMobilnych}
Najbardziej docenianym rodzajem aplikacji mobilnych z punktu widzenia biznesu są aplikacje klienckie. W przypadku tych aplikacji szalenie istotny jest problem połączenia z systemami zewnętrznymi. Wyróżniamy tutaj dwa tryby pracy aplikacji:

\begin{itemize}
\item Tryb online -- urządzenie udostępnia przynajmniej jeden typ komunikacji bezprzewodowej (Wi-fi, Bluetooth, łącza podczerwieni (IrDa), GPRS). W trybie online aplikacje mobilne mają bezpośredni dostęp do zewnętrznych i zdalnych źródeł danych, innych urządzeń mobilnych lub stacjonarnych systemów komputerowych.   
\item Tryb offline -- urządzenie ma bezpośredni dostęp tylko do lokalnie przechowywanych informacji. Dane te mogą być jednak synchronizowane z innymi urządzeniami w czasie krótkich sesji komunikacyjnych. Wymiana danych podczas synchronizacji może następować w obu kierunkach. 
\end{itemize}

%---------------------------------------------------------------------------
\section{Cechy specyficzne dla aplikacji mobilnych }
\label{sec:cechyAplikacjiMobilnych}
Opisując aplikacje mobilnie nie można zapomnieć o cechach specyficznych, które są szalenie ważne z punktu widzenia projektanta i programisty aplikacji mobilnych. To właśnie dzięki tym cechom aplikacje mobilne posiadają tę a nie inną specyfikę:

\begin{itemize}
\item Ograniczone zasoby sprzętowe -- urządzenie przenośnie pod względem zasobów sprzętowych nigdy nie zastąpią komputerów stacjonarnych a tym bardziej urządzeń serwerowych. Właśnie ze względu na tę cechę aplikacje mobilne ukierunkowane są na rozwiązywanie małych - jednostkowych problemów. Projektant aplikacji mobilnych powinien szczególnie zwrócić uwagę na ten aspekt w przypadku projektowania interfejsu do komunikacji z systemami zewnętrznymi, aplikacje mobilne powinny wykorzystywać lekkie interfejsy komunikacyjne typu REST, aby jak najmniej obciążać kartę sieciową urządzenia. 
\item Uproszczony interfejs użytkownika dostosowany do ekranów dotykowych -- cecha ta jest bardzo ważna głównie dla projektanta interfejsu użytkownika, ale ma również znaczący wpływ na rodzaj zadań rozwiązywanych za pomocą tych aplikacji.  
\item Przerwy w działaniu aplikacji -- aplikacje mobilne w przeciwieństwie do innych rodzajów systemów są szczególnie narażone na przerwy w działaniu, najprostszym przykładem może być rozmowa telefoniczna przychodząca w trakcie pracy z aplikacją. Najczęściej w takich przypadkach aplikacje przechodzą w tryb uśpienia i nie ma pewności czy praca zostanie wznowiona. Programiści powinny pamiętać o zapamiętywaniu poszczególnych stanów aplikacji aby użytkownik nie tracił wykonanej pracy. 
\item Dostęp do GEO lokalizacji -- aplikacje mobilne oprócz ograniczeń posiadają również wiele cech dodatkowych wyróżniających je wśród innych systemów, jedną z nich jest dostęp do GEO lokalizacji.  Funkcjonalność ta bardzo szeroko wykorzystywana jest przez systemu zwane Context Aware Middleware. 
\item Dostęp do aparatu/kamery -- jest to kolejna cecha rozszerzająca możliwości aplikacji mobilnych, może być wykorzystana np. jako narzędzie do skanowania kodów kreskowych, albo sposób dokumentacji pracy wykonanej przez użytkownika systemów mobilnych. 
\item Specyficzna architektura dla różnych systemów operacyjnych -- bardzo wiele cech aplikacji mobilnych jest wspólna w przypadku wszystkich systemów operacyjnych dostępnych na rynku, trzeba mieć jednak świadomość, że jest również wiele cech specyficznych dla różnych platform mobilnych. 
\end{itemize}

%---------------------------------------------------------------------------
\section{Komunikaty Push}
\label{sec:push}
Ze względu na zmienne warunki w dostępie do mediów komunikacyjnych, aplikacji mobilne wykorzystują w głównej mierze komunikacje pull, (czyli pobieranie informacji na żądanie) z systemami zewnętrznymi. Najpopularniejsze platformy mobilne jak Android, iOS oraz Windows Phone udostępniają również alternatywny sposób komunikacji - push. 

Komunikacja push odbywa się jednak zawsze za pośrednictwem serwisów udostępnianych przez platformy mobilne do tego celu. Komunikacja ta posiada wiele ograniczeń a już na pewno najważniejszym z nich jest brak pewności że komunikat zostanie w ogóle dostarczony. Komunikacja push powinna być zatem wykorzystywana jedynie do notyfikacji użytkownika o jakimś zdarzeniu a nie do przesyłania informacji kluczowych dla działania aplikacji. 
\chapter{Warstwa pośrednia w zastosowaniu }
\label{cha:example}

Najlepszym sposobem weryfikacji każdego rozwiązania jest jego konfrontacja z rzeczywistością, przez znalezienie realnych przykładów jego użycia. Nie inaczej jest w przypadku proponowanego w niniejszej pracy rozwiązania. 

Pomysłem na przykład jest w tym przypadku proces sprzątania pokoju hotelowego. Celem przykładu jest zademonstrowanie działania warstwy pośredniej do komunikacji procesu biznesowego z aplikacjami mobilnymi, ale również pokazania sposobu integracji takiego rozwiązania z gotowymi systemami zewnętrznymi. 

%---------------------------------------------------------------------------

\section{Przedstawienie koncepcji}
\label{sec:concept}

Prezentacja przykładu zostanie rozpoczęta od przedstawienia koncepcji rozwiązania w postaci diagramów zaprojektowany za pomocą BPMN 2.0. Przedstawione zostaną dwa diagramy z których pierwszy jest sytuacją w której rozpoczyna się właściwy proces sprzątania hotelu. 

\begin{figure}[h]
\centerline{\includegraphics[scale=0.4]{hotelCheckOutProcess}}
\caption{Diagram BPMN procesu wymeldowania klienta z pokoju hotelowego.}
\label{fig:hotelCheckOutProcess}
\end{figure}

Na rysunku ~\ref{fig:hotelCheckOutProcess}, przedstawiono proces wymeldowania gościa hotelowego z pokoju po zakończonym pobycie. Klient hotelu po opuszczeniu pokoju udaje się do recepcji w celu oddania kluczy i uregulowania płatności. W recepcji przebywa recepcjonista, który odpowiedzialny jest za kontakt z klientem, po otrzymaniu kluczy prosi klienta o zapłatę kosztów pobytu. Po dokonaniu płatności klient hotelowy opuszcza hotel, a recepcjonista oznacza pokój hotelowy jako gotowy do posprzątania, rozpoczynając w ten sposób właściwy proces. 

\begin{figure}[h]
\centerline{\includegraphics[scale=0.4]{roomCleanUpProcess}}
\caption{Diagram BPMN procesu sprzątania pokoju hotelowego.}
\label{fig:roomCleanUpProcess}
\end{figure}

Rysunek~\ref{fig:roomCleanUpProcess} przedstawia właściwy proces sprzątania pokoju hotelowego, widać na nim cztery rodzaje aktorów, każdy z nich oznaczony własną linią. Pierwszym z aktorów jest system zarządzania hotelem, jest to miejsce w którym recepcjonista oznacza pokój do posprzątania. System zarządzania hotelem jest miejscem w którym rozpoczyna się proces. Odpowiedzialny jest on za zebranie danych koniecznych do przeprowadzenia sprzątania a następnie przekazanie tych danych do kolejnego aktora, którym jest Serwis Sprzątający. A przypadku realizacji przykładu za pomocą opisanych w niniejszej pracy magisterskiej technologii, aktora tego można utożsamić z procesem BPEL wraz z warstwą pośrednią. Ostani aktor - serwis sprzątający odpowiedzialny jest za przeprowadzenie procesu sprzątania. Proces ten odbywa się przez wysłanie żądania do sprzątaczki (aplikacja mobilna), by po potwierdzeniu posprzątania pokoju wysłać żądanie do nadzorcy (również aplikacja mobilna) który będzie odpowiedzialny za weryfikacje sprzątania.  W przypadku braku akceptacji sprzątania przez nadzorcę pokój będzie musiał być kolejny raz posprzątany. W momencie zatwierdzenia sprzątania, rezultat trafi ponownie do Systemu Zarządzania Hotelem, aby recepcjonista mógł zameldować kolejnych gości. 

Na podstawie opisanego powyżej przykładu wyłonić można następujące systemy, których implementacja zostanie dokładniej opisana w dalszej części pracy.

\begin{itemize}
\item System Zarządzania Hotelem -- Dla celów przykładu, System Zarządzania Hotelem będzie prosta aplikacją internetową odpowiedzialną za udostępnienie formularza dla recepcjonisty oraz wyświetlenie listy rezultatów wykonania procesu.   
\item Proces BPEL -- Będzie to proces zawierający kompletną logikę biznesową odpowiedzialną za przeprowadzenie i weryfikacje sprzątania pokoju hotelowego. Proces ten będzie komunikował się zarówno z warstwą pośrednią jak i z Systemem Zarządzania Hotelem. 
\item Warstwa pośrednia --  Aplikacja internetowa odpowiedzialna za obsługę komunikacji procesu biznesowego z aplikacjami mobilnymi. 
\item Aplikacja mobilna -- aplikacja przeznaczona zarówno dla sprzątaczek jak i dla nadzorców. Udostępniać będzie bardzo prosty interfejs użytkownika służący do wyświetlenia listy zadań oraz do ich przypisywania i rozwiązywania. 
\end{itemize}


%---------------------------------------------------------------------------

\section{System Zarządzania Hotelem }
\label{sec:hotelManagementSystem}

Zadaniem tej aplikacji jest udostępnienie części funkcjonalność Systemu Zarządzania Hotelem przeznaczonej do oznaczania pokoju hotelowego jako wymagającego posprzątania. Aplikacja udostępniać będzie formularz w którym recepcjonista będzie mógł zgłosić pokój do posprzątania. Formularz będzie gromadził taki dane jak numer pokoju, numer piętra oraz kategorię do której należy pokój. Po zatwierdzeniu formularza  dane z niego pochodzące trafią do procesu biznesowego oraz do bazy danych w celu monitorowania postępu procesu. Kiedy proces zakończy swoje działania aplikacja odbierze za pomocą odpowiedniej usługi sieciowej jego rezultat. Odebranie rezultatu skutkować będzie zmianą statusu pokoju. Aplikacja udostępniać będzie również listę pokoi wraz z ich statusami w postaci strony html. 

\subsection{Wybrane technologie}
Aplikacja zostanie zrealizowana przy wykorzystaniu technologii:

\begin{itemize}
\item język programowania Java
\item Spring MVC
\item  Bootstrap Framework -- jest to zbiór  klas css oraz biblioteka stworzona w języku JavaScript, służący do tworzenia interfejsu użytkownika przy pomocy zaawansowanych kontrolek html.
\item Maven
\item Spring Web Services
\item Spring Data oraz Hibernate - jeden z najbardziej popularnych narzędzi ORM, przeznaczonych na platformę Java. 
\item H2 in memory - baza danych której cykl życia jest równoznaczny z cyklem życia aplikacji. Jest ona tworzona podczas uruchamiana aplikacji, dane znajdujące w się w niej są przechowywane w pamięci podręcznej. Bazy tego typu wykorzystywane są przede wszystkim do implementacji testów jednostkowych. Zdecydowanym plusem skorzystania z tego rodzaju bazy danych jest brak konieczności instalacji dodatkowych narzędzi co w przypadku niniejszego przykładu jest dużym udogodnieniem. 
\end{itemize}

\subsection{Implementacja}

\begin{figure}[h]
\centerline{\includegraphics[scale=0.6]{hotelManagementSystemClasses}}
\caption{Diagram klas systemu zarządzania hotelem.}
\label{fig:hotelManagementSystemClasses}
\end{figure}


Na rysunku~\ref{fig:hotelManagementSystemClasses} został przedstawiony diagram klas Systemu Zarządzania Hotelem. Znajdują się na nim jedynie najbardziej istotne klasy z punku widzenia tej implementacji. Klasa MainController jest odpowiedzialna za udostępnienie interfejsu użytkownika oraz za obsługę żądań z niego pochodzących. Udostępnia ona trzy metody: 


\begin{itemize}
\item list -- metoda za pomocą RomRepository pobiera z bazy danych wszystkie pokoje, które zostały zgłoszone do posprzątania. Pobrana lista jest  następnie umieszczana w dynamicznej stronie html za pomocą technologii JSP. 
\item form -- metoda odpowiedzialna za wyświetlenie formularza w postaci html dla użytkownika. 
\item hadleForm -- metoda ta jest uruchamiana w przypadku zatwierdzenia danych wprowadzonych do formularza przez użytkownika. Metoda dodaje nowy pokój do bazy danych oraz uruchamia proces biznesowy z wykorzystaniem klasy CleanUpService. 
\end{itemize}

Druga bardzo ważną klasą jest CleanUpServiceCallbackEndpoint, jest to klasa stworzona z wykorzystaniem technologii Spring Web Services. Klasa ta obsługuje zdefiniowaną w konfiguracji usługę sieciową. Usługa służy do odbierania rezultatów wykonania procesu biznesowego. Jej rolą jest przeczytania komunikatu przysłanego przez proces biznesowy by następnie zmienić status sprzątanego pokoju za pomocą klasy CleanUpService. 

\subsection{Uruchomienie}

Aplikacja jest kompilowana do postaci pliku o rozszerzeniu war. Pliki tego typu mogą być uruchamiana na dowolnym serwerze aplikacyjnym Java, np. Tomcat. Aplikacja powinna zostać uruchomiona na porcie 8181, ponieważ pozostałe aplikacje będą od niej tego wymagały. Np. proces biznesowy swój rezultat będzie wysyłał właśnie do lokalnej maszyny na ten port. 

%---------------------------------------------------------------------------

\section{Proces BPEL }
\label{sec:exampleBPEL}

Proces BPEL jest odpowiedzialny za obsługę logiki biznesowej związanej z przeprowadzeniem procesu sprzątania. Proces służy w pewnym sensie jako narzędzie integracyjne aplikację internetową jaką jest System Zarządzania Hotelem z aplikacjami mobilnymi. 

Przebieg działania procesu rozpoczyna się w Systemie Zarządzania Hotelem, który uruchamia usługę sieciową udostępnianą jako punkt dostępowy procesu. Proces po otrzymaniu żądania z systemu zewnętrznego tworzy nowe zadanie przeznaczone dla osoby sprzątającej pokój by następnie przejść w tryb oczekiwania na jego rezultat. Gdy pokój zostanie posprzątany, proces tworzy kolejne zadanie tym razem przeznaczone dla nadzorcy i ponownie przechodzi w tryb oczekiwania na jego rezultat. Gdy nadzorca zakończy swoje zadanie proces sprawdzi jego rezultat, jeśli sprzątanie pokoju zostało zaakceptowane proces zakończy swoje działanie poprzez wysłanie rezultatu do Systemu Zarządzania Hotelem. Gdy sprzątanie pokoju nie zostanie zaakceptowane proces będzie tworzył  kolejne zadania dla osoby sprzątającej do momentu zatwierdzenia przez nadzorcę. 

\subsection{Implementacja}

\begin{figure}[h]
\centerline{\includegraphics[scale=0.6]{bpelProcess}}
\caption{Wizualizacja procesu BPEL w postaci graficznej za pomocą wtyczki do Eclipsa.}
\label{fig:bpelProcess}
\end{figure}

Do implementacji procesu BPEL została wykorzystana wtyczka do zintegrowanego środowiska programistycznego Eclipse. Wtyczka ta udostępnia narzędzia do modelowania procesów BPEL w sposób graficzny. Na rysunku~\ref{fig:bpelProcess} został przedstawiony zrzut ekranu przedstawiający implementację niniejszego procesu biznesowego. Na rysunku widać doskonale kolejność wywoływania poszczególnych aktywności procesu BPEL, nie widać natomiast zastosowania mechanizmu korelacji oraz zakresu. 

\subsubsection{Korelacja}

Mechanizmem na który w szczególności należy zwrócić uwagę podczas opisu procesu biznesowego jest mechanizm korelacji. Mechanizm ten służy do powiązania dwóch lub więcej operacji wewnątrz jednej instancji procesu. W przypadku niniejszego przykładu proces korelacji wykorzystywany jest do powiązywania operacji stworzenie zadania z operacją odebrania rezultatu zadania. Zazwyczaj środowisko uruchomieniowe posiada więcej niż jedną instancje procesu, w momencie gdy zostaje odebrany rezultat zadania np. sprzątania środowisko powinno w jakiś sposób zidentyfikować instancję procesu do której zaadresowany jest ten rezultat. 

Mechanizm korelacji realizowany jest za pomocą tak zwanych zbiorów korelacji. Zbiór korelacji nie jest niczym innym jak zbiorem zmiennych. Zmienne te inicjalizowane są przy pierwszym użyciu dowolną wartością, podczas kolejnego użycia służą do odszukania instancji procesu o zadanej wartości. 

W niniejszym procesie istnieją dwa zbiory korelacji, zdefiniowane wewnątrz ciała pętli: 

\begin{lstlisting}[caption=Definicja zbiorów korelacji.,numbers=left]
<bpel:correlationSets>
	<bpel:correlationSet name="cleanUpTask" 
			properties="tns:taskUUID"/>
	<bpel:correlationSet name="verifyTask" 
			properties="tns:verifyTaskUUID"/>
</bpel:correlationSets> 


\end{lstlisting}

Zbiory te wykorzystywane są do powiązania operacji stworzenia i odbioru rezultatu zadania sprzątanie i weryfikacja. 

\begin{lstlisting}[caption=Wykorzystanie zbiorów korelacji w aktywności invoke.,numbers=left]
<bpel:receive name="recieveCleanUpTaskResult" partnerLink="client" 
	operation="cleanUpTaskCallback" portType="tns:cleanUpProcess" 
	variable="clientRequest">
            <bpel:correlations>
                <bpel:correlation set="cleanUpTask" initiate="no">
	     </bpel:correlation>    
            </bpel:correlations>
        
</bpel:receive>
\end{lstlisting}

Operacja invoke odpowiedzialna za stworzenie zadania inicjalizuje zbiór korelacji odebraną z warstwy pośredniej wartością unikalnego identyfikatora zadania.

\begin{lstlisting}[caption=Wykorzystanie zbiorów korelacji w aktywności receive.,numbers=left]
<bpel:receive name="recieveCleanUpTaskResult" partnerLink="client" 
		operation="cleanUpTaskCallback"
	          portType="tns:cleanUpProcess" variable="clientRequest">
            <bpel:correlations>
                	<bpel:correlation set="cleanUpTask" initiate="no">
		</bpel:correlation>    
            </bpel:correlations>
</bpel:receive>

\end{lstlisting}

 Operacja recieve następnie na podstawie tego samego unikalnego identyfikatora odnajduje odpowiednią instancje by kontynuować jej działanie. 

\subsubsection{Zakres }
Bezpośrednio powiązany z mechanizmem korelacji jest mechanizm zakresów (ang. scope). Służy on do odizolowania dwóch lub więcej operacji wewnątrz procesu. Izolacja ta może zostać wykorzystana np. do odpowiedniego zarządzania sytuacjami wyjątkowymi. W przypadku niniejszego procesu mechanizm zakresu został zastosowany do odizolowania kolejnych wywołań pętli w celu uzyskania efektu ponownej inicjalizacji zbiorów korelacji w każdym przebiegu pętli. W przypadku nie zastosowania mechanizmy zakresu, podczas pierwszego wywołania przebiegu pętli zbiór korelacji zostałby zainicjalizowany identyfikatorem pierwszego zadania. W przypadku kolejnych wywołań pętli metoda recieve oczekiwała by na identyfikator pierwszego zadania mimo, że zostało ono już rozwiązane a kolejny przebieg pętli stworzył następne zadanie. 

Mechanizm zakresu stosuje się obejmując izolowane elementy w drzewie xml elementem: 

\begin{lstlisting}[caption=Przykład zastosowania zakresu w BPEL.,numbers=left]
<bpel:scope isolated="yes">
...
</bpel:scope>
\end{lstlisting}

\subsection{Uruchomienie}
 Proces BPEL został stworzony zgodnie ze specyfikacją BPEL, może być zatem wdrożony na dowolne środowisko uruchomieniowe dla procesów biznesowych. W przykładzie zastosowano środowisko Apache ODE. 

Apache ODE jest również aplikacją internetową napisaną w języku Java i może być uruchomiony z wykorzystaniem dowolnego serwera aplikacyjnego. Ważne jest aby silnik Apache ODE został uruchomiony z wykorzystaniem portu 8080. System Zarządzania Hotelem będzie próbował uruchomić proces na lokalnej maszynie pod tym właśnie portem.  


%---------------------------------------------------------------------------


\section{Warstwa pośrednia }
\label{sec:exampleMiddleware}

Najbardziej istotną częścią przykładu z punktu widzenia niniejszej pracy magisterskiej jest warstwa pośrednia. Jest ona odpowiedzialna za odbiór  żądań od procesu biznesowego i przekazanie ich do aplikacji mobilnych. Warstwa pośrednia udostępnia w tym celu dwie usługi sieciowe. Jedną do utworzenia zadania - sprzątanie pokoju i drugą do utworzenia zadania - weryfikacja sprzątania. Żądania odbierane przez usługi sieciowe zapisywane są następnie w dokumentowej bazie danych i oczekują na realizacje przez użytkowników mobilnych. Aplikacje mobilne komunikują się warstwą pośrednią za pomocą REST API. Są w stanie w ten sposób zautoryzować się, pobrać listę zadań i rozwiązywać je. Zadania po rozwiązaniu odsyłane są do procesów biznesowego z wykorzystaniem odpowiednich usług sieciowych. 

\subsection{Wybrane technologie}
Warstwa pośrednia z racji tego, że wykorzystuje przedstawioną w niniejszej pracy magisterskiej bibliotekę, została zrealizowana w technologii Spring. I korzysta z bazy danych Mongo DB.

\subsection{Implementacja}

Jak wspomniano wcześniej warstwa pośrednia wykorzystuje opisaną w niniejszej pracy magisterskiej bibliotekę stworzoną do tego celu. Opis implementacji zaczniemy od dwóch najważniejszych plików konfiguracyjnych tej biblioteki. Pierwszym z nich jest plik bpel4mobile.properties znajdujący się w katalogu resources. W pliku tym znajdują się dane do komunikacji z bazą danych MondoDB, oraz ścieżka do drugiego pliku konfiguracyjnego - humanInteractions.xml. Plik xml jest najważniejszym plikiem konfiguracyjnym z punktu widzenia stworzonej biblioteki, zawiera od definicje zadań, grup użytkowników i przypisania poszczególnych grup do zadań. 

\begin{lstlisting}[caption=Plik konfiguracyjny z opisem zadań warstwy pośredniej.,numbers=left]
<?xml version="1.0" encoding="UTF-8"?>
<htd:humanInteractions
	xmlns:htd="http://docs.oasis-open.org/bpel4people/ws-humantask"
	xmlns:xsi="http://www.w3.org/2001/XMLSchema-instance"
	xmlns:xsd="http://www.w3.org/2001/XMLSchema"
	xsi:schemaLocation="http://docs.oasis-open.org/
	bpel4people/ws-humantask 
	http://docs.oasis-open.org/bpel4people/ws-humantask.xsd">

	<htd:logicalPeopleGroups>
		<htd:logicalPeopleGroup name="cleaningLadies">
			<htd:parameter name="correspondingFloor"
				type="xsd:int"/>
		</htd:logicalPeopleGroup>
		<htd:logicalPeopleGroup name="supervisors" />
	</htd:logicalPeopleGroups>
	<htd:tasks>
		<htd:task name="cleanUpTask">
			<htd:priority>5</htd:priority>
			<htd:peopleAssignments>
				<htd:potentialOwners>
					<htd:from 
					logicalPeopleGroup="cleaningLadies">
					<htd:argument 
						name="correspondingFloor">
						eq:request/room/floor
					</htd:argument>
					</htd:from>
				</htd:potentialOwners>
			</htd:peopleAssignments>
		</htd:task>
		<htd:task name="verifyTask">
			<htd:priority>5</htd:priority>
			<htd:peopleAssignments>
				<htd:potentialOwners>
				<htd:from
				logicalPeopleGroup="supervisors" />
				</htd:potentialOwners>
			</htd:peopleAssignments>
		</htd:task>
	</htd:tasks>
</htd:humanInteractions>
\end{lstlisting}

Na powyższym fragmencie kodu widoczny jest pliku konfiguracyjny niniejszego przykładu. Linijki 10-6 zawierają definicję dwóch grup użytkowników, jednej dla sprzątaczek i drugiej dla nadzorców. Sprzątaczki posiadają dodatkowy atrybut którym jest numer piętra na którym pracują. W linijkach 17-41 znajdują się definicję zadań, widać na nich że zadanie sprzątania przypisane jest do sprzątaczek pracujących na danym piętrze, natomiast weryfikację mogą przeprowadzać dowolni nadzorcy. Sposób konfiguracji zadań jest bardzo prosty, przejrzysty i elastyczny. 

Kolejną ważną częścią implementacji warstwy pośredniej jest udostępnienie usług sieciowych dla procesu biznesowego. Zadanie to zrealizowane zostaje przy pomocy modułu Spring WS. Spring WS do tworzenia usług sieciowych preferuje podejście interfejs najpierw (ang. contract first), w celu udostępnienia usług sieciowej konieczne jest zdefiniowanie pliku ze schematem komunikatów xml. Powstają w tej sposób dwa pliki dla dwóch różnych usług sieciowych. Pliki te wykorzystane zostaną do wygenerowanie odpowiednich kontraktów WSDL przez Spring WS. Po stronie kodu Java, powstają odpowiedniki opisanych komunikatów w postaci klas. Po dwie klasy na usługę sieciową do obsługi wiadomości wejściowej i rezultatu. Przykładowo dla weryfikacji sprzątania klasy będą wyglądać następująco: 

\begin{lstlisting}[caption=Klasy zawierające komunikaty wejściowe i wyjściowe usługi weryfikacji sprzątania.,numbers=left]
@XmlRootElement(name="request", 
	namespace= XMLNamespace.VERIFY)
public class VerifyRequest {

	@XmlElement(name="deadline",
		namespace=XMLNamespace.VERIFY)
	private Date deadline;

	@XmlElement(name="cleanUpPerformer", 
		namespace=XMLNamespace.VERIFY)
	private String cleanUpPerformer;

	@XmlElement(name="room", 
		namespace=XMLNamespace.VERIFY)
	private Room room;
}

@XmlRootElement(name="result", 
	namespace="http://bpel4mobile.com/example/hotel/schemas")
public class VerifyResponse {

	public enum Status {
		success, toRepeat
	}

	@XmlElement(name="status", 
		namespace="http://bpel4mobile.com/example/hotel/schemas")
	private Status status;
}
\end{lstlisting}

Spring WS oprócz przygotowania plików ze schematem xml komunikatów, wymaga zdefiniowanie servletu w plikach xml. Servlet ten wskazuje na pliki ze schematem xml oraz definiuje przestrzeń nazw i port usługi sieciowej. 

\begin{lstlisting}[caption=Servlet usługi weryfikacji sprzątania.,numbers=left]
<beans ... >
	<sws:dynamic-wsdl id="verifyService" 
		portTypeName="verifyServicePort" 
		locationUri="http://localhost:8282/hotel-cleanup
			-mobile-middleware/ws/verifyService" 
		targetNamespace="http://bpel4mobile.com/
			schemas/example/verifyService"> 
	<sws:xsd location="/WEB-INF/schema/verify-request.xsd"/> 
	</sws:dynamic-wsdl>
</beans>
\end{lstlisting}

Ostatnią rzeczą do implementacji w celu udostępnienia usługi sieciowej jest przygotowanie odpowiedniej klasy Java która będzie punktem końcowym zdefiniowanego wcześniej servletu. Klasa ta będzie uruchamiać odpowiednią instancję serwisu do obsługi zadań zdefiniowanych w warstwie pośredniej. Dalszą obsługą zadania wraz z odesłaniem odpowiedzi zajmie się przygotowana w poprzednim rozdziale biblioteka. 

\begin{lstlisting}[caption=Punkt końcowy usługi weryfikacji sprzątania.,numbers=left]
@Endpoint
public class VerifyServiceEndpoint {

	@Autowired
	@Qualifier(value="verifyTaskService")
	private TaskService tasService;

	@PayloadRoot(namespace = XMLNamespace.VERIFY, 
		localPart = "VerifyTaskRequest") 
	@ResponsePayload
	public Element handleHolidayRequest(@RequestPayload Element request)
			throws Exception {
		return tasService.handleTaskRequest(request, 
				VerifyRequest.class, 
				VerifyResponse.class);
	}
}
\end{lstlisting}

Do zakończenia implementacji warstwy pośredniej konieczne jest jeszcze dostarczenie informacji o użytkownikach systemu. Biblioteka utworzona w poprzednim rozdziale przewiduje do tego celu odpowiedni interfejs, który zostanie zaimplementowany. Na potrzeby niniejszego przykładu nie konieczne jest tworzenie adaptera do pobierania danych o użytkownikach z bazy danych czy innych systemów katalogowych, utworzymy prostą klasę dostarczającą statyczne dane. 

\begin{lstlisting}[caption=Dostawca informacji o użytkownikach. =,numbers=left]
@Service
public class UserDataProviderImpl 
		extends AbstractUserDataProvider {
	private Map<String, UserData> users = 
		new HashMap<String, UserData>();
	@PostConstructor
	public void init(){
		UserGroupData cleanUpGroupDataFloor1 = 
			new UserGroupData();
		cleanUpGroupDataFloor1.setName("cleaningLadies");
		cleanUpGroupDataFloor1.getArguments()
			.put("correspondingFloor", 1);
		UserGroupData cleanUpGroupDataFloor2 = 
			new UserGroupData();
		cleanUpGroupDataFloor2.setName("cleaningLadies");
		cleanUpGroupDataFloor2.getArguments()
			.put("correspondingFloor", 2);
		UserGroupData supervisorGroup = new UserGroupData();
		supervisorGroup.setName("supervisors");
		UserData jadzia = new UserData();
		jadzia.setUsername("Jadzia");
		jadzia.getGroups().add(cleanUpGroupDataFloor1);
		users.put("Jadzia", jadzia);
		UserData stasia = new UserData();
		stasia.setUsername("Stasia");
		stasia.getGroups().add(cleanUpGroupDataFloor2);
		users.put("Stasia", stasia);
		UserData zdzislaw = new UserData();
		zdzislaw.setUsername("Zdzislaw");
		zdzislaw.getGroups().add(supervisorGroup);
		users.put("Zdzislaw", zdzislaw);
	}
	@Override
	public boolean authenticate(String username, 
				String password) {
		return (users.contains(username) && 
				"password".equals(password));
	}

	@Override
	public UserData getUserData(String username, 
			String password) {
		return users.get(username);
	}
}
\end{lstlisting}

\subsection{Uruchomienie}

Aplikacja budowana jest za pomocą narzędzia maven. Plikiem wynikowym podobnie jak w przypadku systemu zarządzania hotelem jest plik z rozszerzeniem .war, który można uruchomić na dowolnym serwerze aplikacyjnym java. Warstwa pośrednia aby współgrała z pozostałymi systemami, musi zostać uruchomiona na porcie 8282. 

%---------------------------------------------------------------------------

\section{Aplikacja mobilna}
\label{sec:exampleMobileApp}

Aplikacja mobilna jest częścią przykładu odpowiedzialną za rozwiązywanie zadań poprzez udostępnienie użytkownikom odpowiedniego interfejsu. Aplikacja komunikuje się w warstwą pośrednią za pomocą REST API w celu pobrania listy zadań, operacji na tych zadaniach i wysłania rezultatu. Aplikacja została zaimplementowana na najbardziej popularną platformę mobilną - Android. 

Aplikacja wykorzystuje specyficzny dla tej platformy mechanizm synchronizacji zarządzany przez system operacyjny. Synchronizacja uruchamiana jest bez konieczności działania aplikacji w momencie gdy urządzenie ma dostęp do sieci. Przez synchronizacje w tym wypadku rozumiane jest pobranie listy zadań dostępnych dla użytkownika i ich zapis w lokalnej bazie danych (SQLite). Aplikacja przewiduje możliwość logowania więcej jak jednego użytkownika, dane każdego z nich zapisywane są do osobnej bazy danych. Użytkownik po wejściu do aplikacji widzi listę dostępnych dla niego zadań mimo, że może nie mieć dostępu do sieci. Użytkownik może zadeklarować chęć wykonania zadania oraz w momencie gdy jest do niego przypisany rozwiązać je klikając odpowiednie przyciski.

\subsection{Wybrane technologie}

Aplikacja została zrealizowana z wykorzystaniem język Java, platformy Android, bibliotek Spring for Android oraz OrmLite. 

\subsection{Implementacja}

Implementacja aplikacji mobilnej bierze pod uwagę dwie kwestie. Pierwszą z nich jest udostępnienie interfejsu użytkownika do logowania, wyświetlenia listy zadań i ich szczegółów. Kwestia druga to synchronizacja listy zadań z wykorzystaniem interfejsu przygotowanego do tego celu przez platformę android. Wykorzystanie API do synchronizacji gwarantuje możliwość pracy z zadaniami w trybie offline. 

\begin{figure}[h]
\centerline{\includegraphics[scale=0.7]{activityFlowDiagram}}
\caption{Ekrany w aplikacji mobilnej wraz z przepływem komunikacji.}
\label{fig:activityFlowDiagram}
\end{figure}

Na rysunku~\ref{fig:activityFlowDiagram} przedstawiono przepływ komunikacji pomiędzy ekranami aplikacji. Po starcie domyślnie uruchamiany jest ekran logowania, który po poprawnym zalogowaniu otwiera główny ekran aplikacji czyli listę zadań. Z listy zadań w zależności od wybranego typu zadania użytkownik przenoszony jest do szczegółów, gdzie może wykonać akcję przypisania do zadania lub jego rozwiązania jeśli jest przypisany. 

\begin{figure}[h]
\centerline{\includegraphics[scale=0.6]{activitiesClassDiagram}}
\caption{Diagram klas odpowiedzialnych za działanie poszczególnych ekranów.}
\label{fig:activitiesClassDiagram}
\end{figure}

Więcej szczegółów implementacyjnych poszczególnych ekranów widać na rysunku~\ref{fig:activitiesClassDiagram}, przedstawia on diagram klas poszczególnych ekranów oraz serwisów z których te klasy korzystają. Ekran logowania korzysta z serwisu do autoryzacji, który autoryzuje użytkownika wysyłając żądanie do warstwy pośredniej. Drugim z serwisów wykorzystywanych przez ekran logowania jest zarządca sesji, który odpowiada za utworzenie sesji użytkownika, w celu przechowywania jego danych dla pozostałych ekranów. Ekran główny wyszukuje listę zadań użytkownika bezpośredni z bazy danych, nie korzysta z innych serwisów. Ekrany szczegółów poszczególnych zadań wykorzystuje zarówno zarządcę sesji jak i serwis przeznaczony do komunikacji z warstwą pośrednią w celu przypisania lub rozwiązania zadania. 

W celu rozwiązania problemu synchronizacji listy zdań, skorzystano z interfejsów udostępnianych przez platformę android. Każda aplikacji przeznaczona na tą platformę posiada główny plik konfiguracyjny zwany AndroidManifest.xml. W pliku tym zostały wskazane dwa serwisy implementujące interfejs serwisu udostępniony przez platformę, odpowiedzialne za udostępnienie adaptera do synchronizacji i implementacje zarządcy kont użytkowników. 
\begin{figure}[h]
\centerline{\includegraphics[scale=0.6]{androidSyncClasses}}
\caption{Diagram klas odpowiedzialnych za synchronizację.}
\label{fig:androidSyncClasses}
\end{figure}

Rysunek~\ref{fig:androidSyncClasses} przedstawia diagram wyżej wymienionych klas. Cykl życia adaptera synchronizacji, udostępnionego przez odpowiedni serwis, zarządzany jest przez system operacyjny. Proces synchronizacji uruchamiany jest cyklicznie na podstawie danych statystycznych o ilości zsynchronizowanych rekordów zwracanych przez adapter. Synchronizacja uruchamiana jest tylko w momencie dostępu aplikacji do Internetu, bez konieczności jej uruchamiania. Dzięki takiemu działaniu adaptera lista zadań aktualizowana jest mimo że użytkownik nie korzysta z aplikacji i jest zawsze aktualna kiedy użytkownika zechce z niej korzystać mimo braku dostępy do Internetu.

Implementacja zarządcy użytkownikami, jest konieczna do utworzenia kona użytkownika w systemie operacyjnym. Konto to jest wykorzystywane do synchronizacji. 

\subsection{Uruchomienie}

Aplikacja podobnie jak pozostałe systemy jest budowana przy pomocy narzędzia maven. Po poprawnym zbudowaniu aplikacji powstaje plik o rozszerzeniu apk, który może być bezpośrednio zainstalowany na urządzeniu mobilnym z systemem android.


\chapter{Podsumowanie }
\label{cha:podsumowanie}

Celem niniejszej pracy magisterskiej było zestawienie technologii Modelowania Procesów Biznesowych z aplikacjami mobilnymi. W wyniku analizy wyżej wymienionych technologii powstała idea integracji tych technologii za pomocą warstwy pośredniej. Idea przełożona została na projekt, który został opisany w niniejszej pracy magisterskiej i implementację. Implementacja ta posłużyła do przygotowania przykładowego systemu, który kompleksowo pokrywa wszystkie części niezbędne do zastosowania warstwy pośredniej w komercyjnych systemach. 


\begin{thebibliography}{1}

%\bibitem{Dil00}
%A.~Diller.
%\newblock {\em LaTeX wiersz po wierszu}.
%\newblock Wydawnictwo Helion, Gliwice, 2000.
%
%\bibitem{Lam92}
%L.~Lamport.
%\newblock {\em LaTeX system przygotowywania dokumentów}.
%\newblock Wydawnictwo Ariel, Krakow, 1992.
%

\bibitem{EAOBPMWeb}
European Association of Business Process Management.
\newblock {\em {Organization home website}}.
\newblock \\\texttt{http://www.eabpm.org/}.


\bibitem{wiBPMA}
Antipodes.
\newblock {\em {What is a Business Process Management system?}}.
\newblock \\\texttt{http://www.antipodes.bg/en/cubes/what\_is\_bpm/}.

\bibitem{OASISweb}
OASIS.
\newblock {\em {OASIS Web Services Business Process Execution Language (WSBPEL) TC}}
\newblock \\\texttt{https://www.oasis-open.org/committees/tc\_home.php?wg\_abbrev=wsbpel}

\bibitem{OASISBPELSpec}
OASIS
\newblock {\em Web Services Business Process Execution Language Version 2.0}.
\newblock OASIS Standard 11 April 2007 

\bibitem{wiao}
Mulesoft.
\newblock {\em {What is application orchestration?}}
\newblock \\\texttt{http://www.mulesoft.org/what-application-orchestration}

\bibitem{bpel4People}
IBM and SAP.
\newblock {\em WS-BPEL Extension for People – BPEL4People}.
\newblock A Joint White Paper by IBM and SAP July 2005 

\bibitem{rest}
Patryk Yarpo Jar
\newblock {\em {REST – ciekawszy sposób na komunikację client-server}}
\newblock \\\texttt{http://www.yarpo.pl/2012/07/29/rest-ciekawszy-sposob-na-komunikacje-client-server/}

\bibitem{springFramework}
Spring community
\newblock  {\em {Spring Framework - home page}}
\newblock \\\texttt{http://projects.spring.io/spring-framework/}

\bibitem{springWS}
Spring community
\newblock  {\em {Spring Framework - Web Services component}}
\newblock \\\texttt{http://projects.spring.io/spring-ws/}

\bibitem{mongodb}
MongoDB community
\newblock  {\em {Mongo DB documentation}}
\newblock \\\texttt{http://docs.mongodb.org/manual/}


\end{thebibliography}


\end{document}
