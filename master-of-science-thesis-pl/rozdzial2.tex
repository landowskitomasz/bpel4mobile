\chapter{Analiza problemu modelowania procesów biznesowych}
\label{cha:analizaModelowaniaProcesowBiznesowych}

Modelowanie procesów biznesowych jest pojęciem  bardzo ogólnym, nie zawiera w sobie żadnych szczegółów technicznych, określa jedynie pewne specyficzne podejście do rozwiązywania problemów informatycznych. W podejściu tym podczas analizy, główny nacisk kładziony jest na wyłonienie procesów występujących w analizowanym problemie. Proces jest tutaj rozumiany jako zbiór następujących w określonej kolejności operacji prowadzących do osiągnięcia konkretnego celu. Najczęściej modelowanie procesów biznesowych jest rozważane w kontekście działania konkretnego przedsiębiorstwa, w którym wyłonienie procesów oraz odpowiednie nimi zarządzenie ma kluczowe znaczenie dla osiągnięcia sukcesu. 

%---------------------------------------------------------------------------

\section{BPM}
\label{sec:bpm}

W literaturze pojęcie modelowania oraz zarządzanie procesami biznesowymi w kontekście przedsiębiorstw występuje pod skrótem BPM, którego angielskie rozwinięcie to Buisness Process Management (pl.  Zarządzanie Procesami Biznesowymi). Idea BPM jest bardzo popularna i szeroko rozwijana w środowisku IT, powstała nawet organizacja pod nazwą \textit{European Association of Buisness Process Managment} zajmująca się rozwojem i promocją BPM.  Na stronie internetowej organizacji znaleźć możemy oficjalną definicje BPM, która mówi, że w skład Zarządzania Procesami Biznesowymi wychodzi:
\begin{itemize}
\item projektowanie,
\item wykonywanie,
\item dokumentacja,
\item pomiar,
\item monitorowanie,
\item kontrola
\end{itemize}
zautomatyzowanych oraz niezautomatyzowanych procesów biznesowych.[1]

W środowisku IT zdarza się, że BPM traktowany jest jako osobna klasa systemów informatycznych, obok ERP, MES, CRM itd. Z punktu widzenia przedstawionej wyżej definicji trudno zgodzić się z takim podejściem, można jednak zauważyć że BPM de facto może zostać wykorzystany do realizacji każdego z wymienionych rodzajów oprogramowania, lub posłużyć jako narzędzie integrujące wyżej wymienione klasy systemów, w celu stworzenia globalnego systemu zarządzania przedsiębiorstwem. [2]


%---------------------------------------------------------------------------

\section{Projektowanie BPM}
\label{sec:projektowanieBPM}


Zdecydowanie jednym z najistotniejszych etapów w tworzeniu systemu opartego o procesy biznesowe jest etap projektowania. Jednym z najbardziej popularnych narzędzi do tego celu jest BPMN (\textit{Buisness Process Modeling Notation}), która doczekała się już dwóch wersji. W kontekście niniejszej pracy magisterskiej notacja BPMN nie jest szczególnie ważna jednak została wspomniana ze względu na jej wykorzystanie w przykładach. Warto zwrócić uwagę  na fakt, że BPMN nie jest w żadnym stopniu implementacją, jest to jedynie sposób wizualizacji procesu za pomocą diagramu w postaci obrazka.



%---------------------------------------------------------------------------

\section{Wykonywanie BPM}
\label{sec:wykonywanieBPM}

Jak zostało wspomniane na samym początku Modelowanie Procesów Biznesowych jest pojęciem ogólnym nie wskazującym konkretnej technologii, która ma posłużyć do wykonania procesów biznesowych. Z definicji można jednak wnioskować, że środowiskiem do wykonywania procesów biznesowych w przedsiębiorstwach na pewno nie powinna być prosta aplikacja desktopowa czy mobilna. Biorąc pod uwagę wachlarz zastosowań BPM nie trudno dojść do wniosku, że najbardziej odpowiednim środowiskiem dla procesów biznesowych są środowiska aplikacji rozproszonych. 
Powyższe stwierdzenie potwierdza przegląd istniejących systemów uruchomieniowych dla procesów biznesowych. W ogromnej większości są to aplikacje webowe. 

%---------------------------------------------------------------------------

\section{Podsumowanie}
\label{sec:podsumowanieBPM}

Z punktu widzenia niniejszej pracy magisterskiej, głównym wnioskiem płynącym z powyższego opisu BPM jest skala problemów rozwiązywanych przez to podejście oraz ich usytuowanie w środowisku aplikacji rozproszonych. 

