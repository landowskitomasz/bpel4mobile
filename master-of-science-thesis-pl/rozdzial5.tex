\chapter{BPEL vs Aplikacje mobilne}
\label{cha:bpelVSmobileApp}

W rozdziałach poprzednich sporo uwagi zostało skupione na temacie modelowania procesów biznesowych. Został również przybliżony temat aplikacji mobilnych. Nadszedł czas aby odnieść się do tematu niniejszej pracy magisterskiej i spojrzeć na zestawienie tych dwóch technologii. 

W postawionym problemie jak zwykle kluczowym jest odpowiednie podejście do tematu, uświadomienie sobie ograniczeń i odpowiednie dopasowanie technologii. Ograniczeń jest tutaj bardzo wiele głównie od strony aplikacji mobilnych.

 Przykładów na wykorzystanie modelowania biznesowego w środowisku aplikacji mobilnych można wymyślić bardzo wiele, tym bardziej w przypadku aplikacji klienckich przeznaczonych dla biznesu. W tym ostatnim, środowisko mimo, że ukierunkowane na niewielkie aplikacje, jest pewnego rodzaju środowiskiem rozproszonym, w którym do tej pory procesy biznesowe sprawdzały się doskonale. 
%---------------------------------------------------------------------------

\section{Adaptacja środowisk uruchomieniowych na platformach mobilnych}
\label{sec:adaptacjaProcesówNaPlatformyMobilne}

Jednym z możliwych sposobów podejścia do tematu może być próba bezpośredniej adaptacji środowisk uruchomieniowych dla procesów biznesowych na platformach mobilnych. Bezpośrednia adaptacja w tym przypadku oznacza próbę uruchomienia jakiegoś lekkiego środowiska uruchomieniowego na smartphonie. Po krótkim zastanowieniu można powiedzieć, że pomysł może być realny w realizacji - zasoby sprzętowe smartphonów są coraz większe, języki programowania wykorzystywane do tworzenia aplikacji mobilnych są identyczne jak języki do tworzenia środowisk uruchomieniowych (Android - język Java, Windows Phone - język \texttt{C\#}). Gdy jednak kontynuujemy przemyślenia bardzo szybko natrafiamy na mur. Jak wspomniano w rozdziale~\ref{cha:analizaAplikacjiMobilnych}, komunikacja aplikacji mobilnych odbywa się przede wszystkim metodą pull, głównie ze względu na zmienne warunki dostępu do mediów komunikacyjnych. W jaki sposób zatem środowisko uruchomieniowe miałby odbierać komunikację z zewnątrz? Jest przecież jeszcze metoda push, dostarczana przez dostawców platform mobilnych. Skorzystanie z niej jest jednak złym pomysłem, ze względu na brak gwarancji dostarczenia komunikatu. Podejście adaptacyjne zatem nie jest dobrym pomysłem. 

%---------------------------------------------------------------------------

\section{Wykorzystanie procesów biznesowych w systemach back end'owych}
\label{sec:integracjaProcesówZAplikacjamiMobilnymi}

Jak wspomniano wyżej najwięcej zastosowań dla procesów biznesowych w kontekście aplikacji mobilnych można dostrzec w aplikacjach klienckich. Należy się zatem zastanowić w jaki sposób działają typowe aplikacje klienckie. Z powodu ograniczeń opisanych w poprzednim akapicie komunikacja zazwyczaj nie odbywa sie na zasadzie Peer-to-Peer, użytkownicy aplikacji klienckich jednak w jakiś sposób wymieniają między sobą informację. Jest to możliwe dzięki aplikacjom centralnym, zazwyczaj internetowym, zwanych systemami back'endowymi. Aplikacje mobilne w takich przypadkach komunikują się jedynie z systemem back end'owym nie posiadając informacji o sobie nawzajem. W tego rodzaju sposobie komunikacji można dostrzec bardzo wiele cech wspólnych z warstwą orkiestracji opisaną w rozdziale~\ref{cha:bpel}. Można zatem wyciągnąć wniosek, że procesy biznesowe w kontekście aplikacji mobilnych zostaną najefektywniej wykorzystane właśnie po stronie systemu back end'owego. 

Podejście to można nazwać podejściem integracyjnym, w przypadku skorzystania z języka BPEL można sobie wyobrazić, że zrealizowany w ten sposób system back end'owy oprócz obsługi żądań aplikacji mobilnych mógłby komunikować się z innymi systemami przy wykorzystaniu usług sieciowych. Stworzony w ten sposób proces mógłby kontrolować przebieg komunikacji między aplikacjami mobilnymi traktując je jednocześnie na równi z innymi systemami.  Niestety kolejny raz mimo sensownego pomysłu występują problemy w realizacji, mało tego ograniczenie kolejny raz jest takie samo. Proces BPEL jak opisano w rozdziale~\ref{cha:bpel}, do komunikacji poszczególnych węzłów wykorzystuje usługi sieciowe udostępniające odpowiednie interfejsy WSDL. Aplikacje mobilne nie są w stanie jednak udostępnić takich usług. 

W tym przypadku istnieje sposób na poradzenie sobie z tym problemem, można stworzyć pewnego rodzaju adapter w postaci aplikacji internetowej, będącej warstwą pośrednią między procesami BPEL a aplikacjami mobilnymi. Taka warstwa pośrednia miałby za zadanie z jednej strony udostępnienie usług sieciowych dla procesów biznesowych, a z drugiej strony realizowała by komunikacje z aplikacjami mobilnymi, w celu przekazania żądań pochodzących od procesów biznesowych. 

Opisanemu powyżej sposóbowi wykorzystania procesów biznesowych w aplikacjach mobilnych została poświęcona niniejsza praca, zostanie on dokładniej opisany w kolejnych rozdziałach.
