\chapter{Wstęp}
\label{cha:wstep}

Wzrost popularności aplikacji mobilnych głównie ze względu na powszechny dostępu do urządzeń przenośnych takich jak smartphone czy tablet, wpłynął na wzmocnienie ich pozycji na rynku IT. W ostatnim czasie powstał szereg firm zajmujących się tworzeniem tylko tego rodzaju systemów informatycznych. Przyglądając się współczesnym trendom w IT nie sposób więc nie zwrócić szczególnej uwagi  właśnie na aplikacje mobilne. Mimo wcześniej wspomnianego wzrostu popularności systemy mobilne ciągle wymagają od programistów korzystania z niskopoziomowych bibliotek i narzędzi. Ze względu na rodzaj problemów rozwiązywanych przez tego typu aplikacje może to być częściowo zrozumiałe. Aplikacje mobilne jednak to nie tylko proste programy uruchamiane w ograniczonym sprzętowo środowisku, twórcy tego rodzaju aplikacji muszą się zmagać z problemami synchronizacji danych, rozproszonej architektury  itd. 
Z drugiej zaś strony wyżej wymienione problemy są już od dawna znane, co więcej posiadają wypracowane schematy ich rozwiązywania. W przypadku aplikacji internetowych powstało mnóstwo frameworków i narzędzi wspierających tworzenie tego rodzaju systemów. Szablonowe rozwiązania pozwoliły natomiast na przeniesienie rozwiązywania problemów na znacznie wyższy poziom niż pisanie kodu aplikacji. Powstały w ten sposób różnego rodzaju systemy uruchomieniowe logiki biznesowej, skalowane i łatwo modyfikowalne. 
Niniejsza praca magisterska jest próbą wykorzystania istniejących środowisk uruchomieniowych dla logiki biznesowej w środowisku aplikacji mobilnych. 

%---------------------------------------------------------------------------

\section{Cel pracy}
\label{sec:celePracy}

Celem niniejszej pracy magisterskiej jest zestawienie technologii  modelowania logiki biznesowej z aplikacjami mobilnymi. Powinno ono prowadzić do wyciągnięcia wniosków dotyczących sensowności oraz sposobu wykorzystania tych dwóch kompletnie różnych technologii razem. 

%---------------------------------------------------------------------------

\section{Struktura pracy}
\label{sec:strukturaPracy}

W rozdziale ~\ref{cha:bpmVSMobileApplications} została zawarta analiza modelowania procesów biznesowych w kontekście aplikacji mobilnych. Rozdział został podzielony na cztery sekcje, które zawierają kolejno:
\begin{itemize}
\item Sekcja ~\ref{sec:analizaModelowaniaProcesowBiznesowych} - Opis technologii modelowania procesów biznesowych.
\item Sekcja ~\ref{sec:analizaAplikacjiMobilnych} - Opis specyfiki aplikacji na platformy mobilne.
\item Sekcja ~\ref{sec:bpel} - Opis BPEL - jednej z implementacji BPM w środowisku usług sieciowych. 
\item Sekcja ~\ref{sec:bpelVSmobileApp} - Zestawienie BPEL z aplikacjami mobilnymi.
\end{itemize}

Rozdział ~\ref{cha:middleware} jest opisem projektu warstwy pośredniej, której zastosowanie zostało zaproponowane w rozdziale poprzednim. Rozdział rozpoczyna się od przedstawienia bardzo poglądowych diagramów projektu, by następnie poprzez kolejne fazy projektowania rozwiązana przejść do bardziej szczegółowych diagramów klas i sekwencji. 

Zwieńczeniem pracy magisterskiej jest rozdział ~\ref{cha:example}, który zawiera kompleksowy przykład zastosowania przedstawionego wcześniej rozwiązania. Rozdział zawiera poglądowy diagram BPMN implementowanego procesu biznesowego i opis czterech podsystemów niezbędnych do implementacji tego procesu.
