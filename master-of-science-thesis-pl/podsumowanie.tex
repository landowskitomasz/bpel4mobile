\chapter{Podsumowanie }
\label{cha:podsumowanie}

Celem niniejszej pracy magisterskiej była analiza możliwości adaptacji środowisk uruchomieniowych na aplikacjach mobilnych. Cel ten został zrealizowany poprzez przedstawienie podstawowej analizy dwóch wyżej wymienionych technologii, oraz próbę ich zestawienia. W niniejszej pracy magisterskiej zostały zaproponowane dwa różne podejścia do problemu. Pierwsze z nich - podejście adaptacyjne okazało się błędne ze względu na charakterystykę aplikacji mobilnych. Drugie podejście - integracyjne okazuje się znakomitym pomysłem w połączeniu z technologią BPEL. Głębsza analiza rozwiązania skutkuje powstaniem idei wykorzystania warstwy pośredniej odpowiedzialnej za adaptacje specyficznej charakterystyki aplikacji mobilnych do współpracy z procesami biznesowymi w środowisku usług sieciowych. Idea przełożona została na projekt, który został opisany w dalszych rozdziałach niniejszej pracy magisterskiej i implementację. Implementacja ta posłużyła do przygotowania przykładowego systemu, który kompleksowo pokrywa wszystkie części niezbędne do zastosowania warstwy pośredniej w~komercyjnych systemach. Przedstawiony system oprócz przykładu zastosowania traktować można jako wytyczne projektowaniu tego rodzaju aplikacji. Dostrzec w nim można jak w praktyce funkcjonuje podejście top-down w projektowaniu systemów rozproszonych za pomocą procesów biznesowych. 

Wynikiem pracy magisterskiej okazała się być biblioteka klas umożliwiająca zastosowanie przedstawionej koncepcji integracji procesów biznesowych z aplikacjami mobilnymi w komercyjnych zastosowaniach. Biblioteka ta w celu lepszej dystrybucji zadań, która powinna przełożyć się na skuteczniejszej rozwiązywanie problemów może zostać zintegrowania z systemami typu Contex Aware Middleware. Wykonanie takiej integracji możliwe jest przez zastosowanie odpowiednich interfejsów których implementacja może zostać dostosowana do potrzeb konkretnego rozwiązania. 

Biblioteka ponadto może zostać rozszerzona o dodatkowe funkcjonalności jak eskalacja w przypadku zwlekania z rozwiązywaniem zadań oraz notyfikacje o zadaniach do rozwiązania. 
